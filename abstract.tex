\chapter*{Sommario} % senza numerazione
\label{abstract}

\addcontentsline{toc}{chapter}{Sommario} % da aggiungere comunque all'indice

--- soweego automatically links existing Wikidata items about people to trusted identifier catalogs.
With soweego, our beloved knowledge base gets in sync with a giant volume of external information, and gets ready to become the universal linking hub of open data. ---


  Sommario è un breve riassunto del lavoro svolto dove si descrive l'obiettivo, l'oggetto della tesi, le 
metodologie e le tecniche usate, i dati elaborati e la spiegazione delle conclusioni alle quali siete arrivati.  

Il sommario dell’elaborato consiste al massimo di 3 pagine e deve contenere le seguenti informazioni:
\begin{itemize}
  \item contesto e motivazioni 
  \item breve riassunto del problema affrontato
  \item tecniche utilizzate e/o sviluppate
  \item risultati raggiunti, sottolineando il contributo personale del laureando/a
\end{itemize}




