%%%%%%%%%%%%%%%%%%%%%%%%%%%%%%%%%%%%%%%%%%%%%%%
%
% Template per Elaborato di Laurea
% DISI - Dipartimento di Ingegneria e Scienza dell’Informazione
%
% update 2015-09-10
%
% Per la generazione corretta del 
% pdflatex nome_file.tex
% bibtex nome_file.aux
% pdflatex nome_file.tex
% pdflatex nome_file.tex
%
%%%%%%%%%%%%%%%%%%%%%%%%%%%%%%%%%%%%%%%%%%%%%%%

% formato FRONTE RETRO
\documentclass[epsfig,a4paper,11pt,titlepage,twoside,openany]{book}
\usepackage{epsfig}
\usepackage{plain}
\usepackage{setspace}
\usepackage[paperheight=29.7cm,paperwidth=21cm,outer=1.5cm,inner=2.5cm,top=2cm,bottom=2cm]{geometry} % per definizione layout
\usepackage{titlesec} % per formato custom dei titoli dei capitoli

% Pacchetti aggiunti
\usepackage[hyphens]{url}
\usepackage[breaklinks, colorlinks=true, citecolor=red, linkcolor=black, urlcolor=black]{hyperref}
\usepackage{graphicx}
\usepackage[font=small,labelfont=bf]{caption}
\usepackage{subfig}

%%%%%%%%%%%%%%
% supporto lettere accentate
%
%\usepackage[latin1]{inputenc} % per Windows;
\usepackage[utf8x]{inputenc} % per Linux (richiede il pacchetto unicode);
%\usepackage[applemac]{inputenc} % per Mac.

\singlespacing

\usepackage[russian, english]{babel}

\begin{document}

  % nessuna numerazione
  \pagenumbering{gobble} 
  \pagestyle{plain}

\thispagestyle{empty}

\begin{center}
  \begin{figure}[h!]
    \centerline{\psfig{file=logo_unitn_black_center.eps,width=0.6\textwidth}}
  \end{figure}

  \vspace{2 cm} 

  \LARGE{Dipartimento di Ingegneria e Scienza dell’Informazione\\}

  \vspace{1 cm} 
  \Large{Corso di Laurea in\\
    Informatica
    %Informatica
    %Ingegneria dell'Informazione e delle Comunicazioni
    %Ingegneria dell'Informazione e Organizzazione d'Impresa
    %Ingegneria Elettronica e delle Telecomunicazioni
  }

  \vspace{2 cm} 
  \Large\textsc{Elaborato finale\\} 
  \vspace{1 cm} 
  \Huge\textsc{Soweego\\}
  \Large{\it{solid catalogs and weekee go together}}


  \vspace{2 cm} 
  \begin{tabular*}{\textwidth}{ c @{\extracolsep{\fill}} c }
  \Large{Supervisore} & \Large{Laureando}\\
  \Large{Andrea Passerini}& \Large{Massimo Frasson}\\
  \Large{Marco Fossati}\\
  \end{tabular*}

  \vspace{2 cm} 

  \Large{Anno accademico 2017/2018}
  
\end{center}



  \clearpage
 
%%%%%%%%%%%%%%%%%%%%%%%%%%%%%%%%%%%%%%%%%%%%%%%%%%%%%%%%%%%%%%%%%%%%%%%%%%
%%%%%%%%%%%%%%%%%%%%%%%%%%%%%%%%%%%%%%%%%%%%%%%%%%%%%%%%%%%%%%%%%%%%%%%%%%
%% Nota
%%%%%%%%%%%%%%%%%%%%%%%%%%%%%%%%%%%%%%%%%%%%%%%%%%%%%%%%%%%%%%%%%%%%%%%%%%
%% Sezione Ringraziamenti opzionale
%%%%%%%%%%%%%%%%%%%%%%%%%%%%%%%%%%%%%%%%%%%%%%%%%%%%%%%%%%%%%%%%%%%%%%%%%%
%%%%%%%%%%%%%%%%%%%%%%%%%%%%%%%%%%%%%%%%%%%%%%%%%%%%%%%%%%%%%%%%%%%%%%%%%%
  \input{ringraziamenti}
  \clearpage
  \pagestyle{plain} % nessuna intestazione e pie pagina con numero al centro

  
  % inizio numerazione pagine in numeri arabi
  \mainmatter

%%%%%%%%%%%%%%%%%%%%%%%%%%%%%%%%%%%%%%%%%%%%%%%%%%%%%%%%%%%%%%%%%%%%%%%%%%
%%%%%%%%%%%%%%%%%%%%%%%%%%%%%%%%%%%%%%%%%%%%%%%%%%%%%%%%%%%%%%%%%%%%%%%%%%
%% Nota
%%%%%%%%%%%%%%%%%%%%%%%%%%%%%%%%%%%%%%%%%%%%%%%%%%%%%%%%%%%%%%%%%%%%%%%%%%
%% Si ricorda che il numero massimo di facciate e' 30.
%% Nel conteggio delle facciate sono incluse 
%%   indice
%%   sommario
%%   capitoli
%% Dal conteggio delle facciate sono escluse
%%   frontespizio
%%   ringraziamenti
%%   allegati    
%%%%%%%%%%%%%%%%%%%%%%%%%%%%%%%%%%%%%%%%%%%%%%%%%%%%%%%%%%%%%%%%%%%%%%%%%%
%%%%%%%%%%%%%%%%%%%%%%%%%%%%%%%%%%%%%%%%%%%%%%%%%%%%%%%%%%%%%%%%%%%%%%%%%%

    % indice
    \tableofcontents
    \clearpage
    
    
          
    % gruppo per definizone di successione capitoli senza interruzione di pagina
    \begingroup
      % nessuna interruzione di pagina tra capitoli
      % ridefinizione dei comandi di clear page
      \renewcommand{\cleardoublepage}{} 
      \renewcommand{\clearpage}{} 
      % redefinizione del formato del titolo del capitolo
      % da formato
      %   Capitolo X
      %   Titolo capitolo
      % a formato
      %   X   Titolo capitolo
      
      \titleformat{\chapter}
        {\normalfont\Huge\bfseries}{\thechapter}{1em}{}
        
      \titlespacing*{\chapter}{0pt}{0.59in}{0.02in}
      \titlespacing*{\section}{0pt}{0.20in}{0.02in}
      \titlespacing*{\subsection}{0pt}{0.10in}{0.02in}
      
      % sommario
      \clearpage
\newpage
\mbox{~}
\clearpage
\newpage

\chapter*{Summary} % senza numerazione
\label{abstract}
Wikipedia and \textbf{Wikidata} are open source collaborative projects with the goal of collecting and sharing general knowledge. While the former is intended for humans, the latter caters for \textbf{machine-readable data}. Both are not meant to contain \textit{original research}.\footnote{\url{https://en.wikipedia.org/wiki/Wikipedia:No_original_research}}\footnote{\url{https://www.wikidata.org/wiki/Wikidata:What_Wikidata_is_not}} Original research means that no reliable or published material exists referring to reported facts, allegations or ideas.
Such design decision entails that Wikipedia and Wikidata content should be supported by references to external sources.
Nevertheless, \texttt{Wikidata} suffers from a \textbf{lack of references}.
The \textbf{soweego} project aims at mitigating the issue by \textbf{linking} \texttt{Wikidata} to a set of trustworthy target \textbf{catalogs}, a task that can be cast as a record linkage problem.
As a result, \textbf{references and new content} may be mined from these catalogs after the linking.

\textbf{Record linkage}\footnote{\url{https://en.wikipedia.org/wiki/Record_linkage\#Probabilistic_record_linkage}} must cope with several challenges: inaccurate data, heterogeneous data precision, and ambiguity, just to name the key ones.
% Homonyms are the very first issue to deal with, as much as people's second names notation (extended, short, missing).
These challenges can be addressed by comparing other data attributes, such as dates and locations. However, there is no guarantee that \texttt{Wikidata} and the target will share any attribute. \texttt{soweego} addresses the issue by working on a common set of attributes, but for the sake of generalization it supports target-specific ones.

This thesis details a set of deterministic techniques for record linkage: \textit{perfect full name match}; \textit{perfect name with birth and death dates match}; \textit{perfect cross-catalog link match}; \textit{tokenized cross catalog link match}; \textit{normalized full names match}.
Specifically, we designed them to work on the common set of attributes, and we evaluate the results obtained picking the \texttt{MusicBrainz} as target database.
We observe that \textit{URLs to external sources} and \textit{dates} attributes provide a relatively high precision.
Hence, the output produced by these techniques has been added to \texttt{Wikidata}.

Our work on the aforementioned techniques represents the fundamental building blocks and knowledge to scale up from deterministic algorithms to probabilistic ones.
We summarize our contributions to the \texttt{soweego} project as follows.
\begin{itemize}
    \item Analysis of the long tail of candidate target catalogs;
    \item development of the facility to import the \texttt{MusicBrainz} dump in the system;
    \item implementation of the baseline techniques for \texttt{MusicBrainz};
    \item performance evaluation;
    \item software packaging in a portable environment;
    \item intervention to all technical discussions.
\end{itemize}

\pagebreak


%%%%%%%%%%%%%%%%%%%%%%%%%%%%%%%%%%%%%%%%%%%%%%%%%%%%%%%%%%%%%%%%%%%%%%%%%%
%%%%%%%%%%%%%%%%%%%%%%%%%%%%%%%%%%%%%%%%%%%%%%%%%%%%%%%%%%%%%%%%%%%%%%%%%%
%% Nota
%%%%%%%%%%%%%%%%%%%%%%%%%%%%%%%%%%%%%%%%%%%%%%%%%%%%%%%%%%%%%%%%%%%%%%%%%%
%% Sommario e' un breve riassunto del lavoro svolto dove si descrive 
%% l’obiettivo, l’oggetto della tesi, le metodologie e 
%% le tecniche usate, i dati elaborati e la spiegazione delle conclusioni 
%% alle quali siete arrivati.
%% Il sommario dell’elaborato consiste al massimo di 3 pagine e deve contenere le seguenti informazioni: 
%%   contesto e motivazioni
%%   breve riassunto del problema affrontato
%%   tecniche utilizzate e/o sviluppate
%%   risultati raggiunti, sottolineando il contributo personale del laureando/a
%%%%%%%%%%%%%%%%%%%%%%%%%%%%%%%%%%%%%%%%%%%%%%%%%%%%%%%%%%%%%%%%%%%%%%%%%%
%%%%%%%%%%%%%%%%%%%%%%%%%%%%%%%%%%%%%%%%%%%%%%%%%%%%%%%%%%%%%%%%%%%%%%%%%%      
      
      %%%%%%%%%%%%%%%%%%%%%%%%%%%%%%%%
      % lista dei capitoli
      %
      % \input oppure \include
      %
      \clearpage
\newpage
\mbox{~}
\clearpage
\newpage

\chapter{Introduction}
\label{cha:intro}
In 2018, the English Wikipedia has been visited 207.83 billions of times, a 4.73\% increase year by year.\footnote{\url{https://stats.wikimedia.org/v2/\#/en.wikipedia.org/reading/total-page-views/normal|bar|1-Year|~total}} This shows that Wikipedia is one of the key tools for enhancing people knowledge, due to the easily accessible free information. However, the unstructured nature of its content does not enable straightforward machine processing.
% For this reason, projects like Wikidata were created.
Wikidata\footnote{\url{https://www.wikidata.org/}} acts as central storage for structured data of its Wikimedia sister projects, including Wikipedia, Wikivoyage, Wikisource, and others.

Data quality is a crucial factor for the trustworthiness of Wikipedia and Wikidata content. In fact, Wikipedia provides mature citation guidelines to enforce high quality standard.\footnote{\url{https://en.wikipedia.org/wiki/Wikipedia:Citing_sources}}
Likewise, Wikidata allows to store references to (ideally authoritative) sources along with its data.

Nevertheless, \textbf{less than a quarter} of the Wikidata knowledge base (KB) statements currently has a reference to \textbf{non-wiki} sources, and roughly an \textbf{half} of them is totally \textbf{unreferenced}.\footnote{\url{https://docs.google.com/presentation/d/1XX-yzT98fglAfFkHoixOI1XC1uwrS6f0u1xjdZT9TYI/edit?usp=sharing}, slides 15 to 19}
The problem can be alleviated in several ways: we could encourage the community to focus on the referencing task. Another option could be the alignemnt of Wikidata entities to a set of external databases, which can be automated by software. This is a particularly interesting option because it provides a repeatable process.

\begin{figure}[t]
  \begin{center}
   \includegraphics[height=200px]{images/matchexample.png}
   \captionof{figure}{\texttt{soweego} creates a connection (read performs disambiguation) between entries of the source database (Wikidata) and a target database. Image by \href{https://meta.wikimedia.org/wiki/User:Hjfocs}{\texttt{Hjfocs}}, \href{https://creativecommons.org/licenses/by-sa/4.0/deed.en}{CC BY-SA 4.0}}
   \label{fig:soweegomatch}
  \end{center}
\end{figure}

We define the problem as follows: given a Wikidata entity, find a suitable match in the target database.
This is not a trivial task, with homonyms being the first challenge.
For instance, suppose we would like to know more about the singer \textit{John Smith}.
We type the name into the Wikidata search box: multiple results appear, and it takes a while to find the person we are looking for.
Unfortunately, Wikidata does not hold much information about him and our search moves to another database, like MusicBrainz\footnote{\url{https://musicbrainz.org}}.
We must repeat the same procedure as with Wikidata, and after some digging, we manage to find John Smith the singer. This is a match: the MusicBrainz entity can be linked to Wikidata.
\texttt{soweego} aims at solving the issue at a large scale, via disambiguation of a source Wikidata entity to a target database entry. Figure~\ref{fig:soweegomatch} depicts the solution.

Wikidata entities are designed to store this kind of links. In particular, entities like John Smith are called \textit{items} and each of them is composed of an \textit{identifier}, a \textit{fingerprint}, and a set of \textit{statements}.\footnote{\url{https://www.mediawiki.org/wiki/Wikibase/DataModel/Primer}}
Each statement is broken down into \textit{claims} (i.e., property, value pairs) and optional \textit{references} (i.e., the sources of the factual information).
For instance, John Smith could have the claim (\texttt{given name}, \texttt{John}), together with its reference (\texttt{stated in}, \texttt{Duckburg Municipality archive}).
The identifier links to external sources are expressed as claims too.
The John Smith match between Wikidata and MusicBrainz would be e.g., expressed as a claim over the Wikidata item with identifier \texttt{Q666} (\texttt{MusicBrainz artist ID}, \texttt{77a1c579-3532-491c-86bd-595ddd4780cc}), where the latter value corresponds to the MusicBrainz identifier.

Officially, \texttt{soweego} has the following goals:\footnote{\url{https://meta.wikimedia.org/wiki/Grants:Project/Hjfocs/soweego\#Project_goals}}
\begin{enumerate}
    \item to ensure live maintenance of identifiers for people in Wikidata, via link validation;
    \item to develop a set of linking techniques that align people in Wikidata to corresponding identifiers in external catalogs;
    \item to ingest links into Wikidata, either through a bot (confident links), or mediated by curation (non-confident links);
    \item to achieve exhaustive coverage (ideally 100\%) of identifiers over 4 large-scale trusted catalogs;
    \item to deliver a self-sustainable application that can be easily operated by the community after the end of the project.
\end{enumerate}

The remainder of this thesis is structured as follows. In section \ref{cha:2} we report a brief review of the state of the art. The preliminary analysis of the candidate targets is detailed in section \ref{cha:3}. Section \ref{cha:4} describes the project architecture with a focus on the matching strategies, which we evaluate against MusicBrainz in section \ref{cha:5}. We draw our conclusions in section \ref{cha:conclusioni}.


      \clearpage
\newpage

\chapter{Related work}
\label{cha:2}
The alignment of Wikidata to third-party structured databases may tackle the lack of references, but it is a complex task. Although a given label can be the same between Wikidata and an external database, there could be ambiguous homonyms. To overcome the issue, we need to exploit other attributes in addition to labels. Choosing them is a challenge itself, since Wikidata and the target database have probably different attribute sets. It is not even assumable that the attributes will be the same among all the entities in the same KB, like Wikidata. 

\texttt{SocialLink} is a system that aligns KB entries of people and organizations to the  corresponding social media profiles,\cite{DBLP:conf/semweb/NechaevCG17a} and shares our problem. Its approach is to pick a minimal subset of attributes: name, difference between person and organization and temporal information that tells if the entity is alive or existent. Similarly, we choose full name, birth and death dates, as well as a set of URLs related to the entity. Unlike \texttt{SocialLink}, we allow the addition of target-specific attributes.

The exceeding attributes can improve the linking process, but they can also be exploited in a KB population task. In fact, mapping the semantics of these attributes against the Wikidata ontology would result in the addition of referenced statements. These statements cannot replace the work done by \texttt{StrepHit}\cite{DBLP:journals/semweb/FossatiDG18} or the one described in \cite{self:SocialLink/TypePrediction}, but we still view it as a contribution. In contrast to us, \texttt{StrepHit} focuses on unstructured data, typically free-text documents from Web sources.

\cite{self:SocialLink/Embeddings} exploits an enhanced representation of the social media content, compared to \cite{DBLP:conf/sac/NechaevCG17}. Despite the improvement, we argue that the approach will not be helpful in \texttt{soweego}, since we cannot assume the availability of any social media data.

The alignment task deals with a lot of queries on those attributes, so working with the target database APIs could be an issue: not only API usage is restricted, but Web requests also bring latency in code execution. Similarly to \cite{DBLP:conf/sac/NechaevCG17}, we work with an iternal database, but we populate it through the target dumps, instead of the social medium feed.


      \chapter{Preliminary analysis}
\label{cha:3}
The very first task of this project is to select the target databases.\footnote{\url{https://meta.wikimedia.org/wiki/Grants:Project/Hjfocs/soweego\#Work_package}} We see two directions here: either we focus on a few big and well known targets as per the project proposal, or we can try to find a technique to link a lot of small ones from the long tail, as suggested by \texttt{ChristianKl}\footnote{\url{https://meta.wikimedia.org/wiki/Grants_talk:Project/Hjfocs/soweego\#Target_databases_scalability}}.

We used \textbf{SQID}\footnote{\url{https://tools.wmflabs.org/sqid/\#/browse?type=properties}} as a starting point to get a list of people databases that are already used in Wikidata, sorted in descending order of usage.\footnote{\textit{Select datatype} set to \textit{ExternalId}, \textit{Used for class} set to \textit{human Q5}} This is useful to split the candidates into \textit{big} and \textit{small} fishes, namely the head and the (long) tail of the result list respectively.

Quoting \texttt{ChristianKl}, it would be ideal to create a configurable tool that enables users to add links to \textit{new databases in a reasonable time-frame} (standing for no code writing). Consequently, we carried out the following investigation: we considered as small fishes all the entries in \textbf{SQID} with an external ID data type, used for class  human (Q5)\footnote{\url{https://www.wikidata.org/wiki/Q5}}, and with \textbf{less than 15 uses in statements}. It results that some critical issues need to be solved to follow this direction, as described in the following lines.

The analysis of a small fish can be broken down into a set of steps. This is also useful to translate the process into software and to make each step flexible enough for dealing with the heterogeneity of the long tail targets. The steps have been implemented into a piece of software by \texttt{MaxFrax96}.\footnote{\url{https://github.com/MaxFrax/Evaluation}}

\section{Retrieving the dump}
\label{cha:31}
The naive technique to link two database is, for each entity in the first one, doing a look up into the second one. Since databases contain a lot of data, we need the dumps to avoid APIs restrictions and slow computation due to connection latency.
Moreover we focus on people, it is therefore necessary to obtain the appropriate dump for each small fish we consider.

\subsection{Problem}
\label{cha:311}
In the real world, such a trivial step raises a first critical issue: not all the database websites give us the chance to download the dump.

\subsection{Solutions}
\label{cha:312}
A non technical complexity solution would be contacting the databases administrators and discuss dump releases for Wikidata, however requires a lot of manual work becoming hard to scale up.

On the contrary, building autonomously the dump would scale up much better. Given a valid URI for each entity, we can re-create the dump. However, this is not trivial to generalize: sometimes it is impossible to retrieve the list of entities, sometimes the URIs are merely HTML pages that require Web scraping. For instance \texttt{Welsh Rugby Union men's player ID (P3826)}\footnote{\url{https://www.wikidata.org/wiki/Property:P3826}}, \texttt{Berlinische Galerie artist ID (P4580)}\footnote{\url{https://www.wikidata.org/wiki/Property:P4580}}, \texttt{FAI ID (P4556)}\footnote{\url{https://www.wikidata.org/wiki/Property:P4556}}, at the time of writing, need scraping for both the list of entities and each entity; \texttt{Debrett's People of Today ID (P2255)}\footnote{\url{https://www.wikidata.org/wiki/Property:P2255}}, \texttt{AGORHA event identifier (P2345)}\footnote{\url{https://www.wikidata.org/wiki/Property:P2345}}, at time of writing, do not seem to expose any list of people.

\section{Mapping to Wikidata}
\label{cha:32}
The long tail is roughly broken down as follows:
\begin{itemize}
    \item XML;
    \item JSON;
    \item RDF;
    \item HTML pages with styling and whatever a Web page can contain.
\end{itemize}

\subsection{Problem}
\label{cha:321}
Formats are heterogeneous. We focus on open data and RDF, as dealing with custom APIs is out of scope for this investigation. We also hope that the open data trend of recent years would help us. However, a manual scan of the small fishes yielded poor results. There were 56 targets in our long tail, out of \textbf{16} randomly picked candidates, only \texttt{YCBA agent ID}\footnote{\url{https://www.wikidata.org/wiki/Property:P4169}} was in RDF, and has thousands of uses in statements at the time of writing this report.
\subsection{Solution}
\label{cha:322}
To define a way (by scripting for instance) to translate each input format into a standard project-wide one. This could be achieved during the next step, namely ontology mapping between a given small fish and Wikidata.

\section{Handling the format}
\label{cha:33}
Linking Wikidata items to target entities requires a mapping between both meta-data/schemas.

\subsection{Solution}
\label{cha:331}
The mapping can be manually defined by the community: a piece of software will then apply it. To implement this step, we also need the common data format described above.

\subsection{Side note: available entity meta-data}
\label{cha:332}
Small fishes may contain entity meta-data which are likely to be useful for automatic matching. The entity linking process may dramatically improve if the system is able to mine extra property mappings. This is obvious when meta-data are in different languages, but in general we cannot be sure that two different databases hold the same set of properties, if they have some in common.

\section{Conclusion}
\label{cha:34}
It is out of scope for the project to perform entity linking over the whole set of small fishes. On the other hand, it may make sense to build a system that lets the community plug in new small fishes with relative ease. Nevertheless, this would require a reshape of the original proposal, which comes with its own risks:
\begin{itemize}
\item it is probably not a safe investment of resources;
\item eventual results would not be in the short term, as they would require a lot of work to create a flexible system for everybody's needs;
\item it is likely that the team is not facing eventual extra problems in this phase.
\end{itemize}

Most importantly, a system to plug new small fishes \textbf{already exists}. \textbf{Mix'n'match}\footnote{\url{https://tools.wmflabs.org/mix-n-match/}} is specifically designed for the task.\footnote{\url{http://magnusmanske.de/wordpress/?p=471}}. Instead of reinventing the wheel, we joined efforts with our advisor \texttt{Magnus Manske}\footnote{\url{https://meta.wikimedia.org/wiki/User:Magnus_Manske}} in his work on big fishes\footnote{\url{http://magnusmanske.de/wordpress/?p=478}}


      \clearpage
\newpage

\chapter{System architecture}
\label{cha:4}

\begin{figure}[t]
  \begin{center}
   \includegraphics[width=\textwidth]{images/archi2.pdf}
   \captionof{figure}{\texttt{soweego} architecture overview}
   \label{fig:architecture}
  \end{center}
\end{figure}

\texttt{soweego} is a pipeline that performs several tasks in order to link entities between Wikidata and the target database. We drive the description by means of our work on \texttt{MusicBrainz}.\footnote{\url{https://musicbrainz.org}} \texttt{MusicBrainz} is a community-maintained open-source database of music information.\footnote{\url{https://musicbrainz.org/doc/About}} Given our people use case, we focus on the subset of musical artists.

\section{Importer}
\label{cha:41}
The importer module is in charge of downloading the latest target database dump and store it in an internal database table. The importer is designed to slice the dump in order to only keep the desired subset. Since \texttt{soweego} processes different databases, a configurable middleware takes care of the task. The middleware allows to pre-process the data and to map it against our extendable model. The core model is as follows:
\begin{itemize}
    \item \textit{Catalog ID}, i.e., the identifier of the entity in the target database;
    \item \textit{Label}, i.e., the full name;
    \item \textit{Birth Date};
    \item \textit{Birth date precision};
    \item \textit{Death Date};
    \item \textit{Death date precision}.
\end{itemize}
The extendable schema allows to leverage the peculiarities of the targets, while shared procedures can be applied thanks to the core model.

A common issue (in music databases above all) is people having multiple names, known as \textbf{aliases}. The schema easily handles them by treating the aliases as standalone entities. Clearly, these standalone entities duplicate all the original entity data, except the label. Since the pipeline only reads from the database, a de-normalized approach naturally fits better than a normalized one.

The import phase actually attempts to create two tables: one as described above, and one for the \textbf{external links}. External links are valid URLs pointing to additional resources out of the target database. The core model chosen for links is as follows:
\begin{itemize}
    \item \textit{Entity ID}, i.e., a foreign key to retrieve the person information;
    \item \textit{URL}, i.e., the original full link available in the dump;
    \item \textit{Tokens}, i.e., a tokenized version of the link.
\end{itemize}
The \textit{tokens} attribute contains the link without the \textit{meaningless} words of a well-formed URL, such as \texttt{http}. This is inspired by Natural Language Process (NLP) techniques for highlighting the semantics of a text. For instance, the tokenized link does not distinguish the same resource referenced by \texttt{m.facebook.com} or \texttt{facebook.com}.


\section{Linker}
\label{cha:42}
The linker module is the core engine of the whole system: it is in charge of linking Wikidata to the chosen target database. At the time of writing this thesis, it does not exploit any statistical/probabilistic approach, such as neural networks. It is instead implemented as a set of rule-based linking strategies. A machine learning approach will probably be explored in the further steps. The rule-based linking strategies can be viewed as a set of features for a probabilistic classifier.

\subsection{Perfect name match}
\label{cha:421}
Checking if two human entities share the same full name is the most straightforward linking strategy, but it is prone to many errors. First, it fails at homonyms disambiguation; second, it is not tolerant against string differences. For instance, people with a second name like \textit{George Walker Bush} can appear also as \textit{George Bush} or \textit{George W. Bush}. On top of that, the label \textit{George Bush} is ambiguous because it can refer to both \textit{George Bush junior} or \textit{George Bush senior}.
To reduce the amount of ambiguous entities, we apply this strategy to a relevant source database subset. In the case of \texttt{MusicBrainz}, we slice the musicians subset out of Wikidata. As observed in the example above, this heuristic does not completely solve the problem: for instance, both the \textit{Bushes} were politicians.

\subsection{Perfect name with birth and death dates match}
\label{cha:422}
The perfect name match strategy can be improved by involving other attributes in the process. The \textit{George Bush} ambiguity described above is easily solved through the birth date attribute. However, the availability of dates can be an issue. In fact, dates can be missing or with low precision (e.g., month or day may be missing). Therefore, the dates match can only deal with the least precise shared date.
Some math built on the \textit{birthday problem}\footnote{\url{https://en.wikipedia.org/wiki/Birthday_problem}} gives us an idea of how likely is to have homonyms with same full birth date.\footnote{\url{https://www.capgemini.com/2011/09/same-name-same-birth-date-how-likely-is-it/}}
Nevertheless, we expect that our use case databases do not fall into this problem, since they do not cover all existing people.

\subsection{Cross-database URL match}
\label{cha:423}
As widely mentioned in this thesis, multiple Web sources may represent the same entity (humans in our case). In general, there is a one-to-many relationship between an entity and its representations in the Web, such as a Twitter profile, a MusicBrainz page, a Wikidata item, etc.
Indeed, we can consume the low-hanging fruits of such multiple representation at linking time: as a rule of thumb, two database entries can be linked if they share an URL pointing to the same resource. In some optimal cases, MusicBrainz can even contain Wikidata entity URLs.

Wikidata does not explicitly expose external links: it is rather a combination of an \texttt{external ID} and a \texttt{URL formatter}. A valid reference to the external source can be built from these data.

The naïve linking strategy is a perfect string match between computed Wikidata URLs and the target database ones. However, it is not robust to URL heterogeneity: for instance, it fails when one URL starts with \texttt{http} and the other with \texttt{https}. Making a Web request to these two URLs and comparing the response would be the most suitable way to assess their equality. Still, it would bears a significant latency in code execution.

We improve the strategy robusness by taking inspiration from NLP techniques.
The intuition is that there are many variable parts in a valid URL: for instance, \texttt{http} versus \texttt{https}, the presence or absence of a trailing slash \texttt{/}, of leading \texttt{www} or \texttt{m}, and so forth.
Consquently, we tokenized the URL and removed \textit{stop words} like the aforementioned ones. For instance, \url{https://twitter.com/GeorgeHWBush} would become \texttt{['twitter', 'com', 'GeorgeHWBush']}.

\subsection{Normalized full names match}
\label{cha:424}
As described in Section~\ref{cha:422}, the perfect name match strategy has some flaws. For instance, \textit{Vladimir Vladimirovič Putin} or \textit{vladimir vladimirovič putin} would not match, due to the different usage of capital letters. Another issue arises when comparing different language alphabets, such as Latin \textit{Vladimir Vladimirovič Putin} and Cyrillic \begin{otherlanguage}{russian}\textit{Владимир Владимирович Путин}\end{otherlanguage}.
There are many normalization techniques that help reduce the impact of these problems. The first problem is easily solved by always working with lowercased strings; the second one requires transliteration\footnote{\url{https://en.wikipedia.org/wiki/Transliteration}} of non-Latin alphabets into Latin via conventional conversion tables. In addition, diacritics get normalized to the ASCII character set, thus solving mismatches due e.g., to accents omission.
Finally, word tokenization takes place, resulting in a set of tokens from the normalized string. Tokenization is implemented as a string split by non-word characters, specificially through the regular expression \verb@\W+@.

\section{Ingestor}
\label{cha:43}
The main goal of the system is to improve Wikidata content: hence. confident output should be directly added. Wikidata bots\footnote{\url{https://www.wikidata.org/wiki/Wikidata:Bots/}} are non-human users allowed to perform high-volume edits. A bot account must first undergo community discussion for eventual approval, since it can damage a lot of data.
Specifically, \texttt{soweego} adds confident output to Wikidata through a bot included in the ingestor module, while we plan to upload non-confident output to the \texttt{Mix'n'match}\footnote{\url{https://tools.wmflabs.org/mix-n-match/}} tool, which enables human curation of identifier matches.
The first approved task\footnote{\url{https://www.wikidata.org/wiki/Wikidata:Requests_for_permissions/Bot/soweego_bot}} of the \texttt{soweego bot}\footnote{\url{https://www.wikidata.org/wiki/User:Soweego_bot}} is the addition of links in the form of \texttt{External ID}\footnote{\url{https://www.wikidata.org/wiki/Wikidata:Glossary/en\#External_identifier}} \textit{claims},\footnote{\url{https://www.wikidata.org/wiki/Wikidata:Glossary/en\#Claim}} or \textit{references}\footnote{\url{https://www.wikidata.org/wiki/Wikidata:Glossary/en\#Reference}} whenever the claim already exists.


\section{Validator}
\label{cha:44}
Given a target database, the validator module retrieves already linked Wikidata entities and performs validation checks according to a set of criteria. First, dead or wrong identifiers should be removed. More specifically, the target database may not contain the entity anymore or the entity may have moved to another identifier. In these cases, the validator interacts with the ingestor to remove the invalid link.
In the subsequent run, the linker module will also propose matches to those entities that were affected, thus potentially fixing eventual validation errors.

The second validation criterion relies on a set of essential attributes, namely gender, birth date, birth place, death date, and death place. Clearly, a check over this criterion can only run if both the Wikidata and the target database entities share at least a subset of these attributes. In case of mismatches, the link is marked as invalid and the validator sends it to the ingestor for deprecation.

We foresee the implementation of more fine-grained checks focusing on data consistency. For instance, a dead person cannot be married after the death date.
The validator module will play a central role in the final system: first, it will serve as a tool to detect divergence between Wikidata and a target database; second, it will be responsible of the training set construction for linking strategies based on machine learning.


      \clearpage
\newpage

\chapter{Results}
\label{cha:5}
\texttt{soweego} has been tested against the 19 September 2018 \texttt{MusicBrainz} dump\footnote{\url{ftp://ftp.eu.metabrainz.org/pub/musicbrainz/data/fullexport/}} and a sample of Wikidata entities.
The MusicBrainz artist subset contains \textbf{986,765 artists}\footnote{Entries in MusicBrainz tagged as artists or not tagged} and \textbf{181,221 aliases}. The Wikidata sample consists of \textbf{1100} entities randomly picked from 1\% of all musicians with no MusicBrainz link.\footnote{Entities having an \textit{occupation} property (\texttt{P106}) with value \textit{musician} (\texttt{Q639669}) and all its direct subclasses}
The table below shows the performance of the linking strategies outlined in Section~\ref{cha:42}.

\begin{table}[h]
    \centering
    \caption{Baseline linking strategies performance evaluation}
    \begin{tabular}{| l | c | c | c |}
        \hline
        \textbf{Strategy} & \textbf{Total matches} & \textbf{Checked} & \textbf{Precision} \\ \hline
        Perfect names & 381 & 38(10\%\%) & 84,2\%\\ \hline
        Perfect names + dates & 32 & 32 (100\%) & 100\%\\ \hline
        Normalized names & 227 & 24(10,6\%) & 70,8\%\\ \hline
        Perfect Link & 71 & 71(100\%) & 100\% \\ \hline
        Token Link & 102 & 102(100\%) & 99\%\\ \hline
    \end{tabular}
\end{table}

We first observe that link-based strategies achieve the best performance. They match almost 10\% of the sample with near 100\% precision. Figure~\ref{fig:linksvenn} illustrates why their coverage is 10\%: the \textit{perfect link matches} are a subset of the \textit{token link matches}.
We note that link-based strategies may be improved if we also consider the \texttt{official home page} Wikidata property, which has not been used so far.

The perfect name strategy improves recall with a 35\% coverage over the Wikidata sample, at the cost of precision. We assessed a randomly picked 10\% of these matches, obtaining a precision of 84\%. We argue that the relatively high precision stems from the very high quality of MusicBrainz data and the small size of the checked set.
We also isolated the subset of matches obtained by the perfect name strategy alone and not by other ones (cf. the leftmost set in Figure~\ref{fig:perfectsandtlinkvenn}), resulting in a precision of 68\%.\footnote{10\% of the leftmost set considered, 19 valid on 28 tests}

\begin{figure}[h]
  \begin{center}
   \includegraphics[height=200px]{images/plots/links.png}
   \captionof{figure}{Venn diagram of the links matches}
   \label{fig:linksvenn}
  \end{center}
\end{figure}

\begin{figure}[h]
  \begin{center}
   \includegraphics[height=200px]{images/plots/perfname_tlink_perfnamedate.png}
   \captionof{figure}{Venn diagram of perfect name matches, token link matches and perfect name and dates matches}
   \label{fig:perfectsandtlinkvenn}
  \end{center}
\end{figure}

The normalized name strategy performs similarly to the perfect name one. This is possibly due to the relatively big intersection of matches between the two strategies, as shown in Figure~\ref{fig:perfectandnormvenn}.
We notice that the normalized name strategy has a worse recall than the perfect name one. After further investigation, we realized that Wikidata multilingual labels are responsible for such performance decrease: the current implementation actually adds them to the token set of the entity and we trigger a match only if the source and target token sets are equal.
Hence, Arabic, Chinese, or Russian labels that are only available in Wikidata and not in MusicBrainz create a consistent amount of false negatives.

Unsurprisingly, date-based strategies yield high precision, although we note that they are not sufficient alone. Therefore, we can view them as constraints for other strategies, such as in the perfect name and dates one.

\begin{figure}[h]
  \begin{center}
   \includegraphics[height=200px]{images/plots/names.png}
   \captionof{figure}{Venn diagram of the perfect name matches and the normalized names matches}
   \label{fig:perfectandnormvenn}
  \end{center}
\end{figure}


      \chapter{Conclusion}
\label{cha:conclusioni}
\texttt{Wikidata} is a general-purpose structured knowledge base housed by the Wikimedia Foundation.\footnote{\url{https://wikimediafoundation.org/}}
Its community is becoming more and more active in data curation: in the last year, the number of edits increased by 75.26\%, compared to the previous year.\footnote{\url{https://stats.wikimedia.org/v2/\#/wikidata.org/contributing/edits/normal|bar|1-Year|~total}} Despite such increased effort, Wikidata still suffers from a lack of references that support the trustworthiness of its content: currently, only less than \textbf{a quarter} of it has a reference to \textbf{non-wiki} sources and roughly a \textbf{half} is totally \textbf{unreferenced}.\footnote{\url{https://docs.google.com/presentation/d/1XX-yzT98fglAfFkHoixOI1XC1uwrS6f0u1xjdZT9TYI/edit?usp=sharing}, slides 15 to 19}

In this thesis, we described the first development iteration of \texttt{soweego}, an automatic linking system for large catalogs that aims at filling the reference gap.
Specifically, we illustrated the  \texttt{Wikidata} - \texttt{MusicBrainz} use case.
Our contribution boils down to a set of core building blocks, the most prominent being the \textit{linker}, plus the \textit{ingestor}, \textit{validator}, and \textit{importer} components, which lay the foundation for the final product.
Despite the project is only 3 months old, with a planned duration of 1 year, we managed to add \textbf{several hundreds of high-quality matches}\footnote{\url{https://www.wikidata.org/wiki/Special:Contributions/Soweego_bot}} to \texttt{Wikidata}.

In the \textit{linker} module, we explored rule-based matching strategies, which serve as a baseline system for construction and comparison  of future machine learning-based strategies.
Furthermore, we acquired insights on the actual data quality of \texttt{Wikidata} and earned expertise on the key features for improvement of the linking process.
As described in Section \ref{cha:5}, links are a very precise feature to rely upon. Although dates perform well, they are not a valid standalone feature: on the other hand, they have a positive impact on the precision if they are combined to other ones.
Finally, full name matching is a risky feature, as expected, creating a lot of false positives.

The very next steps will focus on the improvement of existing strategies.
First, the \textit{cross-database URL} strategy introduced in Section \ref{cha:423} will consume \textit{official homepages} of \texttt{Wikidata} entities, as well as the \textit{ISNI code} of the \texttt{MusicBrainz} ones. Moreover, we will investigate how tokenization in \textit{normalized names} strategy (cf. Section \ref{cha:424}) can be better exploited.

Future work will include deeper research in the record linkage literature to improve the system, with a special attention on probabilistic approaches.
Machine learning could be an interesting direction to take, as shown in \cite{DBLP:conf/semweb/NechaevCG17a}.
The immediate goal is to obtain a confidence score for each computed match, expressed as probability.
The baseline strategies could become our features in a machine learning-based solution.
Moreover, inspired by \cite{DBLP:conf/semweb/NechaevCG17a}, we plan to use the full names strategies to select a subset of entities as input for the linking phase.
In this way, our system would perform faster, thanks to the lower number of entities to check. This is possible just because we verified that full names matches tend to be a super set of the other strategies matches.

\clearpage
\newpage
\mbox{~}
\clearpage
\newpage
      %\input{capitolo4}
      
      
    \endgroup


    % bibliografia in formato bibtex
    %
    % aggiunta del capitolo nell'indice
    \addcontentsline{toc}{chapter}{Bibliografia}
    % stile con ordinamento alfabetico in funzione degli autori
    \bibliographystyle{plain}
    \bibliography{biblio}
%%%%%%%%%%%%%%%%%%%%%%%%%%%%%%%%%%%%%%%%%%%%%%%%%%%%%%%%%%%%%%%%%%%%%%%%%%
%%%%%%%%%%%%%%%%%%%%%%%%%%%%%%%%%%%%%%%%%%%%%%%%%%%%%%%%%%%%%%%%%%%%%%%%%%
%% Nota
%%%%%%%%%%%%%%%%%%%%%%%%%%%%%%%%%%%%%%%%%%%%%%%%%%%%%%%%%%%%%%%%%%%%%%%%%%
%% Nella bibliografia devono essere riportati tutte le fonti consultate 
%% per lo svolgimento della tesi. La bibliografia deve essere redatta 
%% in ordine alfabetico sul cognome del primo autore. 
%% 
%% La forma della citazione bibliografica va inserita secondo la fonte utilizzata:
%% 
%% LIBRI
%% Cognome e iniziale del nome autore/autori, la data di edizione, titolo, casa editrice, eventuale numero dell’edizione. 
%% 
%% ARTICOLI DI RIVISTA
%% Cognome e iniziale del nome autore/autori, titolo articolo, titolo rivista, volume, numero, numero di pagine.
%% 
%% ARTICOLI DI CONFERENZA
%% Cognome e iniziale del nome autore/autori (anno), titolo articolo, titolo conferenza, luogo della conferenza (città e paese), date della conferenza, numero di pagine. 
%% 
%% SITOGRAFIA
%% La sitografia contiene un elenco di indirizzi Web consultati e disposti in ordine alfabetico. 
%% E’ necessario:
%%   Copiare la URL (l’indirizzo web) specifica della pagina consultata
%%   Se disponibile, indicare il cognome e nome dell’autore, il titolo ed eventuale sottotitolo del testo
%%   Se disponibile, inserire la data di ultima consultazione della risorsa (gg/mm/aaaa).    
%%%%%%%%%%%%%%%%%%%%%%%%%%%%%%%%%%%%%%%%%%%%%%%%%%%%%%%%%%%%%%%%%%%%%%%%%%
%%%%%%%%%%%%%%%%%%%%%%%%%%%%%%%%%%%%%%%%%%%%%%%%%%%%%%%%%%%%%%%%%%%%%%%%%%
    

    %\titleformat{\chapter}
    %    {\normalfont\Huge\bfseries}{Allegato \thechapter}{1em}{}
    % sezione Allegati - opzionale
    %\appendix
    %\input{allegati}

\end{document}
